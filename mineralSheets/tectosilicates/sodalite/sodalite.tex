\documentclass[10pt]{article}
\usepackage{longtable}
\usepackage{float}
\usepackage{wrapfig}
\usepackage{rotating}
\usepackage[normalem]{ulem}
\usepackage{amsmath}
\usepackage{textcomp}
\usepackage{marvosym}
\usepackage{wasysym}
\usepackage{amssymb}
\usepackage{hyperref}
\usepackage{color,soul} % for highlighting
\usepackage{graphicx}
%\graphicspath{{subdir1/}{subdir2/}{subdir3/}...{subdirn/}}
\graphicspath{{/Users/benjaminbass/seacloud/class/earthMaterials/mineralSheets/tectosilicates/sodalite/images/}}

\usepackage{frame,color}
\usepackage{framed}
\usepackage{minibox}

\usepackage[version=3]{mhchem}
%How to Use MChem
% \ce{SO4^2-}
% \ce{^{227}_{90}Th+}
% \ce{A\bond{-}B\bond{=}C\bond{#}D}
% \ce{CO2 + C -> 2CO}
% \ce{SO4^2- + Ba^2+ -> BaSO4 v}

% \usepackage[T1]{fontenc}
% \usepackage{tilting} %bring title up
% \setlength{\droptitle}{-10cm} 


\author{Benjamin Bass}
\date{24 February 2016}
\title{\vspace{-2.0cm}Sodalite} %bring title up temporary Fix

\begin{document}

\maketitle

% \framebox{Use frameboxes until figure out alignmen}

\begin{center}
\includegraphics{sodalite}  
\end{center}

\framebox[15cm][l]{\textbf{General Mineral Formula}: \ce{Na8Al6Si6O24Cl2} }\
\framebox[15cm][l]{\textbf{Mineral Chemical Class}: Techtosilicates }\
\framebox[15cm][l]{\textbf{Specific Gravity}: 2.27-2.33 }\
\framebox[15cm][l]{\textbf{Hardness}: 5.5-6.0 }\
\framebox[15cm][l]{\textbf{Cleavage}: \{110\}Poor. Uneven, Conchoidal Fracture Brittle }\
\framebox[15cm][l]{\textbf{Luster}: Vitreous }\
\framebox[15cm][l]{\textbf{Streak}: White. But the streak of lazurite may be blue.}\
\framebox[15cm][l]{\textbf{Characteristic Color(s)}: Light-Dark Blue; May vary. \hl{May Flourese} }\
\framebox[15cm][l]{\textbf{Crystal System}: Isometric }\
\framebox[15cm][l]{\textbf{Crystal Class}: $\overline{4}3m$ }\

\begin{framed}
  \textbf{Crystal Description (common forms, habit, etc.)}:  Massive, Anhedral. 
\end{framed}

\begin{framed}
  \textbf{Environment (where you find the material}:  \hl{Silicon-deficient, alkali-rich igneous rocks. Forms with feldpathoids.}
\end{framed}


\begin{framed}
  \textbf{Common Mineral Associations (in samples, also consult text, notes}: Lazurite, pyroxene, calcite and other silicates.
\end{framed}

\begin{framed}
  \textbf{Scientific Usage/Significance}: Few. Folk medicine for Diabetes. Example of flourescence. 
\end{framed}

\begin{framed}
  \textbf{Industrial or Social Use/Significance}:  Dimension stone. Used to face
buildings and other decorative purposes. Folk medicine for diabetes.
\end{framed}

\begin{framed}
  \textbf{Environmental Significance}: Very little. Can be found in contact, metamorphosed carbonate rocks.
\end{framed}


% Possible other Solutions
% \framebox(300,20){\minibox{\textbf{R-Sq}:For example}}

\end{document}
%%% Local Variables:
%%% mode: latex
%%% TeX-master: t
%%% End:
