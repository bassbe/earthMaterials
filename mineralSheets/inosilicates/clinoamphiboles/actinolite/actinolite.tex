\documentclass[10pt]{article}
\usepackage{longtable}
\usepackage{float}
\usepackage{wrapfig}
\usepackage{rotating}
\usepackage[normalem]{ulem}
\usepackage{amsmath}
\usepackage{textcomp}
\usepackage{marvosym}
\usepackage{wasysym}
\usepackage{amssymb}
\usepackage{hyperref}
\usepackage{color,soul} % for highlighting
\usepackage{graphicx}
\graphicspath{{/Users/benjaminbass/seacloud/class/earthMaterials/picBank/}}

\usepackage{frame,color}
\usepackage{framed}
\usepackage{minibox}

% \usepackage[T1]{fontenc}
% \usepackage{tilting} %bring title up
% \setlength{\droptitle}{-10cm}

\usepackage[version=3]{mhchem}
% How to Use MChem
% \ce{SO4^2-}
% \ce{^{227}_{90}Th+}
% \ce{A\bond{-}B\bond{=}C\bond{#}D}
% \ce{CO2 + C -> 2CO}
% \ce{SO4^2- + Ba^2+ -> BaSO4 v}


\author{Benjamin Bass}
\date{2 March 2016}
\title{\vspace{-2.0cm}Actinolite} %bring title up temporary Fix

\begin{document}

\maketitle

% \framebox{Use frameboxes until figure out alignmen}

\begin{center}
  \includegraphics[scale=.04]{actinolite1}\footnote{Green, elongated, prismatic crystals. Fibrous}
  \includegraphics[scale=.04]{actinolite2}\footnote{Distinctive Green Color.}
\end{center}



\framebox[15cm][l]{\textbf{General Mineral Formula}: \ce{Ca2(Mg\text{,}Fe)5Si8O22(OH)2} }\
\framebox[15cm][l]{\textbf{Mineral Chemical Class}: Inosilicates : Clinoamphiboles }\
\framebox[15cm][l]{\textbf{Specific Gravity}: 2.9-3.5 }\
\framebox[15cm][l]{\textbf{Hardness}: 5.5-6.0 }\
\framebox[15cm][l]{\textbf{Cleavage}: \hl{2,2 prismatic, irregular fracture. Brittle. Fibrous} }\
\framebox[15cm][l]{\textbf{Luster}: Vitreous, silky }\
\framebox[15cm][l]{\textbf{Streak}: White }\
\framebox[15cm][l]{\textbf{Characteristic Color(s)}: \hl{Light to dark green; grayish green to black} }\
\framebox[15cm][l]{\textbf{Crystal System}:  Monoclinic}\
\framebox[15cm][l]{\textbf{Crystal Class}: 2/\it{m} }\

\begin{framed}
  \textbf{Crystal Description (common forms, habit, etc.)}: \hl{Elongated prismatic crystals in bladed groups, fibrous, and columnar. Also radiates as wheat sheaf formations and can form thin hairlke masses with interlocking fibers.}
\end{framed}

\begin{framed}
  \textbf{Environment (where you find the material)}: \hl{Contact and regional metamorphic rocks, serpentine deposits, hydrothermal replacement deposits, and as a secondary mineral in igneous basalt.}
\end{framed}

\begin{framed}
  \textbf{Common Mineral Associations (in samples, also consult text, notes}: Albite, Barite, Chlorite, Talc, Epidote, Muscovite, Serpentine
\end{framed}

\begin{framed}
  \textbf{Scientific Usage/Significance}: None
\end{framed}

\begin{framed}
  \textbf{Industrial or Social Use/Significance}: One of the two minerals that form the gemstone Jade. Some fibrous varieties used as asbestos.
\end{framed}

\begin{framed}
  \textbf{Environmental Significance}: None
\end{framed}

% Possible other Solutions
% \framebox(300,20){\minibox{\textbf{R-Sq}:For example}}

\end{document}
%%% Local Variables:
%%% mode: latex
%%% TeX-master: t
%%% End:
