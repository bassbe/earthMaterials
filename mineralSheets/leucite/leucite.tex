\documentclass[10pt]{article}
\usepackage{longtable}
\usepackage{float}
\usepackage{wrapfig}
\usepackage{rotating}
\usepackage[normalem]{ulem}
\usepackage{amsmath}
\usepackage{textcomp}
\usepackage{marvosym}
\usepackage{wasysym}
\usepackage{amssymb}
\usepackage{hyperref}
\usepackage{graphicx}
\graphicspath{{/Users/benjaminbass/Desktop/}}

\usepackage{frame,color}
\usepackage{framed}
\usepackage{minibox}

% \usepackage[T1]{fontenc}
% \usepackage{tilting} %bring title up
% \setlength{\droptitle}{-10cm} 


\author{Benjamin Bass}
\date{24 February2016}
\title{\vspace{-2.0cm}Leucite} %bring title up temporary Fix

\begin{document}

\maketitle

% \framebox{Use frameboxes until figure out alignmen}

\framebox[15cm][l]{\textbf{General Mineral Formula}: $KAlSi_{2}O_{6}$ }\
\framebox[15cm][l]{\textbf{Mineral Chemical Class}: Techtosilicates }\
\framebox[15cm][l]{\textbf{Specific Gravity}: 2.45-2.50 }\
\framebox[15cm][l]{\textbf{Hardness}:  }\
\framebox[15cm][l]{\textbf{Cleavage}:  Cleavage }\
\framebox[15cm][l]{\textbf{Luster}:  Luster }\
\framebox[15cm][l]{\textbf{Streak}:  Streak}\
\framebox[15cm][l]{\textbf{Characteristic Color(s)}:  Color }\
\framebox[15cm][l]{\textbf{Crystal System}:  Crystal System }\
\framebox[15cm][l]{\textbf{Crystal Class}:  Crystal Class }\

\begin{framed}
  \textbf{Crystal Description (common forms, habit, etc.)}:  Crystal Description
\end{framed}

\begin{framed}
  \textbf{Environment (where you find the material}:  Environment
\end{framed}

\begin{framed}
  \textbf{Common Mineral Associations (in samples, also consult text, notes}:  Common Associations
\end{framed}

\begin{framed}
  \textbf{Scientific Usage/Significance}:  Scientific Usage/Significance
\end{framed}

\begin{framed}
  \textbf{Industrial or Social Use/Significance}:  Industiral/Social Use
\end{framed}

\begin{framed}
  \textbf{Environmental Significance}:  Environmental Significance
\end{framed}

% \includegraphics{}

% Possible other Solutions
% \framebox(300,20){\minibox{\textbf{R-Sq}:For example}}

\end{document}
%%% Local Variables:
%%% mode: latex
%%% TeX-master: t
%%% End: