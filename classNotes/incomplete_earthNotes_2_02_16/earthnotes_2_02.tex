% Created 2016-02-08 Mon 11:52
\documentclass[11pt]{article}
\usepackage[utf8]{inputenc}
\usepackage[T1]{fontenc}
\usepackage{fixltx2e}
\usepackage{graphicx}
\usepackage{longtable}
\usepackage{float}
\usepackage{wrapfig}
\usepackage{rotating}
\usepackage[normalem]{ulem}
\usepackage{amsmath}
\usepackage{textcomp}
\usepackage{marvosym}
\usepackage{wasysym}
\usepackage{amssymb}
\usepackage{hyperref}
\tolerance=1000
\usepackage[version=3]{mhchem}
\author{Benjamin Bass}
\date{02 February 2016}
\title{Earth Notes}
\pagestyle{fancy}
\hypersetup{
  pdfkeywords={},
  pdfsubject={},
  pdfcreator={Emacs 24.4.1 (Org mode 8.2.10)}}
\begin{document}

\maketitle
\tableofcontents


\section{Review}
\label{sec-1}
Know for the Test
Most Abundent Elements in the Crust:

\begin{center}
\begin{tabular}{|c|c|c|}
\hline
\textbf{Element} & \textbf{Charge} & \textbf{Coordination Number}\\
 &  & (fold coordination)\\
\hline
Si & 4 & 4\\
O & -2 & X\\
Na & 1 & 8\\
Ca & 2 & 6 (or 8)\\
K & 1 & 12\\
Al & 3 & 6/4\\
Mg & 2 & 6\\
Fe & 2/3 & 6\\
\hline
\end{tabular}
\end{center}

\pagebreak

\section{Mineral Composition}
\label{sec-2}

\uline{Mineral Chemical Site Assignment (more examples in p. 198-200)}

ex: Sphalerite (ZnS)

Actual analysis sample that we collected:

\begin{center}
\begin{tabular}{|c|c|c|c|c|}
\hline
Element & Weight Percent & (Grams per Mole) &  & Normalized\\
\hline
Fe & 18.25\% & 55.85 & 0.327 & .312\\
Mn & 2.66\% & 54.94 & 0.048 & .046\\
Cd & .28 & 112.41 & .002 & .002\\
Zn & 44.67 & 65.58 & 0.683 & .651\\
S & 33.57 & 32.07 & 1.047 & 1.0\\
\hline
Total & 99.43\% &  &  & 1.011\\
\hline
\end{tabular}
\end{center}


Final Formula: (Zn$_{\text{.651}}$ Fe$_{\text{.312}}$ Mn$_{\text{.046}}$ Cd$_{\text{.002}}$)S

That is an easy one. Most of the time, when we're dealing with something like Silicates, we have a situation like this:

Silicate Analysis: Olivine \$ \ce\{$^{\text{VI}}$(MgFe)2\} \ce\{ \^{}\{IV\}SiO4\}\$

\ce{SiO2} 40.99

\begin{center}
\begin{tabular}{|c|c|c|c|c|c|c|}
\hline
Elem. & WP & AW & MAP & xOxy & AO & CFP\\
\hline
SiO$_{\text{2}}$ & 40.99 & / 60.08 & .6822 & x2 & 1.3642 & .999\\
FeO & 8.58 & 71.85 & .1194 & x1 & .1194 & .115\\
Fe$_{\text{2}}$O$_{\text{3}}$ & .50 & 159.69 & .0031 & x3 & .0093 & .009\\
MgO & 50.00 & 90.31 & 1.2403 & x1 & 1.2401 & 1.816\\
MnO & .20 & 70.94 & .0028 & x1 & .0028 & .004\\
\hline
Total & 100.27 &  &  &  & 2.732 & \\
\hline
\end{tabular}
\end{center}

\begin{center}
\begin{tabular}{ll}
\hline
Key & Value\\
\hline
Elem. & Element Normalization FActor\\
WP & Weight Percent\\
AW & Atomic Weight Cadion Formula \%\\
MAP & Molecular Atomic Proportion\\
xOxy & xOxygens\\
AO & Atomic Opop(?)\\
NF & Normalization Factor (4.0/2.732)\\
CFP & Cadion Formula Percentage\\
\hline
\end{tabular}
\end{center}

2.732 $\rightarrow$ .999

Normalize everything to 4 Oxygens.

Normalize to 4 oxygens. Normalizing factor is 2.732.

Normalize to exactly 1.0 S.

If done right, all the cadions should add up to 1.


\dfrac{1}{1.047}


Between 99\% and 100\% is a good analysis.

\uline{Final Formula}
 \ce\{(mg1.816Fe.175$^{\text{2+}}$Fe.009$^{\text{3+}}$Mn.04)Si.999O4\}

\subsection{Why Minerals Grow}
\label{sec-2-1}
$\rightarrow$ Depends on P,T,X conditions
\subsubsection{Thermodynamics}
\label{sec-2-1-1}
Used to describe and predict the equillibrium state of a system.

ex: Si, Al, Ca, O

\ce{CaO}\\
\ce{Si2}\\
\ce{Al2O3}\\
\ce{Al2SiO5}\\
\ce{CaSiO3}\\
etc\ldots{}

\uline{Gibbs Free Energy}
One of the ways we can quantify free energy using thermodynamics.

G = f(P, T, X) (x means composition)

Every substance, including minerals, has some $\Delta$ G$_{\text{i}}$ = f(P,T,X)

Stable Equillibrium of a system is the one with the lowest 
<should be sum of some sort sumdeltaG>
\Sum $\Delta$ G
<Insert Picture of Stability graph>

Diamond wants to break down to graphite. 

All this allows us to write REACTIONS

ex: C$_{\text{Diamond}}$ = C$_{\text{Granite}}$

at the surface P,T: $\Delta$ G$_{\text{graphite}}$ \textless{} $\Delta$ G$_{\text{Diamond}}$

ex: CaAl$_{\text{2}}$Si$_{\text{2}}$O$_{\text{8}}$ = CaAl$_{\text{2}}$Si$_{\text{2}}$O$_{\text{8}}$
      anarthite                melt

at surface P,T            $\Delta$ G\_$_{\text{amorthite}}$ < $\Delta$ G$_{\text{anothite}}$
                                              <----------------------
                                                   reaction proceeds

ex: KAl$_{\text{2}}$(AlSi$_{\text{3}}$)O$_{\text{10}}$(OH)$_{\text{2}}$ + SiO$_{\text{2}}$ =   KAlSi$_{\text{3}}$O$_{\text{8}}$  + Al$_{\text{2}}$SiO$_{\text{5}}$ +  H$_{\text{2}}$O
                muscovite              quartz        feldspar         sillimanite


at very high Temp > 700 degrees Celsius     \Sum $\Delta$ G$_{\text{ksparSilliminH20}}$ < \Sum $\Delta$ G$_{\text{muscovitequartz}}$

so Kspasr is stable or reaction -------------------->
                                  reaction proceeds


ex:
     2 KAlSi$_{\text{3}}$O$_{\text{8}}$ + H$_{\text{2}}$O + 2H$^{\text{+}}$ = Al$_{\text{2}}$Si$_{\text{2}}$O$_{\text{5}}$(OH)$_{\text{4}}$ + 2K$^{\text{+}}$ +         4SiO$_{\text{2}}$
        Feldspar         watah   acid        kalimite clay            classified k+    Dissolved Silicon
% Emacs 24.4.1 (Org mode 8.2.10)
\end{document}
