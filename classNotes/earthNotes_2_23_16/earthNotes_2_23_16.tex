% Created 2016-04-02 Sat 23:30
\documentclass[11pt]{article}
\usepackage[utf8]{inputenc}
\usepackage[T1]{fontenc}
\usepackage{fixltx2e}
\usepackage{graphicx}
\usepackage{longtable}
\usepackage{float}
\usepackage{wrapfig}
\usepackage{rotating}
\usepackage[normalem]{ulem}
\usepackage{amsmath}
\usepackage{textcomp}
\usepackage{marvosym}
\usepackage{wasysym}
\usepackage{amssymb}
\usepackage{hyperref}
\tolerance=1000
\usepackage[version=3]{mhchem}
\author{Benjamin Bass}
\date{23 February 2016}
\title{Earth Materials: Techto and Phyllosilicateso}
\hypersetup{
  pdfkeywords={},
  pdfsubject={},
  pdfcreator={Emacs 24.4.1 (Org mode 8.2.10)}}
\begin{document}

\maketitle
\tableofcontents

\pagebreak

\section{Potassium Feldspar (KSpar)}
\label{sec-1}
Not quite as abundant as plagioclase feldspar, but still extremely abundant.
Kind of like quartz.
\subsection{Environment of KSpar}
\label{sec-1-1}
\begin{itemize}
\item Silica Rocks (felsic)
\begin{itemize}
\item granites
\item rbydlites
\end{itemize}
\item High temperature metamorphic rocks (granite)
\item Sedementary (ex: arkose sandstone)
\item Pegmatite
\end{itemize}
\subsection{Chemical Substitutions}
\label{sec-1-2}
\begin{itemize}
\item \ce{Fe^3+} sub in for \ce{Al^3+}, gives it a color characteristic of feldspar :revisit:
\item Pb can sub in green color and produce "Amazonite"
\end{itemize}

\section{Plagioclase Feldspar}
\label{sec-2}
\begin{itemize}
\item its everywhere
\item in more felsic rocks: Na-rich with Low Temperatures
\item in more mafic rocks: Ca-ric with High Temperatures
\end{itemize}

\section{Zeolite}
\label{sec-3}
\begin{itemize}
\item 80 naturally occuring zeolites.
\item All zeolites contain structural water \ce{H2O}
\begin{itemize}
\item structural water is alwasy lost in the formation
\end{itemize}
\item General Formula: $M_{x}D_{y} (Al_{x + 2y} Si_{n-x-2y}O_{2n}) \cdot mH_{2}O$
\begin{itemize}
\item M = \ce\{Na\text{,}K\}
\item D = \ce\{Ca\text{,}Mg\text{,}Ba\}
\item 4-fold site = $(Al_{x + 2y} Si_{n-x-2y}O_{2n})$
\end{itemize}
\item The structure is a linked framework of :revisit: with long tunnels
\item in zeolite:
\begin{itemize}
\item \ce\{Na\text{,}Ca\text{,}H2O\}\ldots{}etc are loosly bound and exchangable in the tunnels.
\item dessecant
\item water quality $\ce{Na^+} \iff \ce{Ca^2+}$
\item contaminate clean up
\end{itemize}
\end{itemize}

\subsection{Common Zeolites}
\label{sec-3-1}
\begin{center}
\begin{tabular}{lllll}
\hline
Amalcine & (Na) & isometric &  & \\
\hline
Chabazite & (Ca,Na,K) & Triclinic &  & \\
Heulandite & (Na,Cu) & Mono &  & \\
Stilbite & (Na,Ca) & Mono & [elongated & blocks]\\
Natrolite & (Na) & Ortho & [radiationg & fibres]\\
\end{tabular}
\end{center}
% Emacs 24.4.1 (Org mode 8.2.10)
\end{document}