% Created 2016-04-02 Sat 19:10
\documentclass[11pt]{article}
\usepackage[utf8]{inputenc}
\usepackage[T1]{fontenc}
\usepackage{fixltx2e}
\usepackage{graphicx}
\usepackage{longtable}
\usepackage{float}
\usepackage{wrapfig}
\usepackage{rotating}
\usepackage[normalem]{ulem}
\usepackage{amsmath}
\usepackage{textcomp}
\usepackage{marvosym}
\usepackage{wasysym}
\usepackage{amssymb}
\usepackage{hyperref}
\tolerance=1000
\usepackage[version=3]{mhchem}
\author{Benjamin Bass}
\date{18 February 2016}
\title{Earth Materials: Intro to Silicates}
\hypersetup{
  pdfkeywords={},
  pdfsubject={},
  pdfcreator={Emacs 24.4.1 (Org mode 8.2.10)}}
\begin{document}

\maketitle
\tableofcontents

\pagebreak

\section{Marching Throught the Silicates}
\label{sec-1}

Most Polymerized -> Least Polymerized.

\subsection{Most Polymerized: Tectosilicates}
\label{sec-1-1}

Make up 2/3rds of the Crust

Simplets =  \ce{SiO2} Group (silica group)

We find that the \ce{SiO2} group has many form
Polymorphs (same chemical form, different groups)

\subsubsection{Silica Polymorphs}
\label{sec-1-1-1}

Alpha Quartz
Coesite

really important environment of growth: quartz

Environment:
\begin{itemize}
\item 2nd most abundant
\item can grow or be found:
\begin{itemize}
\item igneaus
\item Metamorphic
\item sedimentary
\item hydrothermal
\item These need silica saturated chemistry
\item they're Felsic (high in silica)
\end{itemize}

\item Felsic
\item Igneaus rocks
\begin{itemize}
\item granite -> pegmatite (beautiful y economic) (ultra-felsic igneous rock)
\item rhyolite
\item 
\end{itemize}
\item Metamorphic
\begin{itemize}
\item almost any schist, queiss
\item Sedimentary
\begin{itemize}
\item as common and detrital grains
\item as a chemical cement
\end{itemize}
\end{itemize}
\end{itemize}

\uline{Natural Fluids}
\begin{itemize}
\item quartz precipitate
\item can be very fine grained
\begin{itemize}
\item "cryptocrystalline"
\begin{itemize}
\item agate
\item chalcedony
\item chert or flint
\end{itemize}
\end{itemize}
\end{itemize}

Fulgarite: if lightening hits silica-rich soil.

Opal: 
\begin{itemize}
\item Not quite a mineral because its Amorphous
\item SiO$_{\text{2}}$ + H$_{\text{2}}$O
\item low T fluids
\item also biomineral
\item ex: plant phytoliths
\end{itemize}

\textbf{Whats the most abundant mineral in the crust: Feldspar}
\textbf{Quartz is in second place}

Quartz: Crystal Shape, hexagonal prysm with degraded symmetry. Glassy luster.
Color is no good. 
What causes color in Quartz: 
\begin{itemize}
\item clear
\item smokey (bc aluminum. if quartz in invironment
\end{itemize}

\uline{Feldspars}
most abundant in the crust
(Si + Al)

\rule{\linewidth}{0.5pt}
O                = 1/2


Structure
<classPic>

hole = 9-fold distorted site: K, Na, Ca

T$_{\text{1}}$, T$_{\text{2}}$ = Si or Al





Specific Felspar Minerals are distinguished by the 9-fold site Cation
and \uline{Al$_{\text{1}}$, Si content ordering}

\uline{Feldspars}

Feldspar composition

(K,Na)$_{\text{1-x}}$ Ca$_{\text{x}}$ Al$_{\text{1+x}}$ Si$_{\text{3-x}}$ O$_{\text{8}}$
where x = 0 to 1

\textbf{Ternary Diagram}


3 Polymorphs @ Kspar
differ only in their ordering of Al, Si

Sanidine: Complete Disorder -> monoclinic > 900celcius
Orthoclase: somewhere in the middle    -> monoclinic
Microcline: completely ordered -> triclinic (low symmetry) < 500celcius

Plag:
\begin{enumerate}
\item albite
\item oligoclase
\item andesine
\item labradorite
\item bytonite
\item anorthite
\end{enumerate}

recall \uline{exolution}
\begin{itemize}
\item refers to chemical unmixing upon cooling below the \uline{soldus}.
\end{itemize}

perthitic texture in K-spar


\uline{Twinning}
3 types of twins
\begin{enumerate}
\item Contact Twins: shares a plane
\item Interpenetration twin: grown together= might share rotational axis
\item Polysynthetic Twins: many repeated crystals
\end{enumerate}

Twins get pink highlighter.
% Emacs 24.4.1 (Org mode 8.2.10)
\end{document}