% Created 2016-04-06 Wed 03:49
\documentclass[11pt]{article}
\usepackage[utf8]{inputenc}
\usepackage[T1]{fontenc}
\usepackage{fixltx2e}
\usepackage{graphicx}
\usepackage{longtable}
\usepackage{float}
\usepackage{wrapfig}
\usepackage{rotating}
\usepackage[normalem]{ulem}
\usepackage{amsmath}
\usepackage{textcomp}
\usepackage{marvosym}
\usepackage{wasysym}
\usepackage{amssymb}
\usepackage{hyperref}
\tolerance=1000
\usepackage[version=3]{mhchem}
\author{Benjamin Bass}
\date{\today}
\title{em\_notes\_2\_23}
\hypersetup{
  pdfkeywords={},
  pdfsubject={},
  pdfcreator={Emacs 24.4.1 (Org mode 8.2.10)}}
\begin{document}

\maketitle
\tableofcontents

Earth Materials


\section{Continuing Feldspars\textit{<2016-02-23 Tue>}}
\label{sec-1}


includes 
\begin{itemize}
\item granites
\item rhylites
\end{itemize}
\section{Techtosilicates}
\label{sec-2}
\begin{itemize}
\item You can get KSpar in metamorphic rocks, but that's more specialized
\begin{itemize}
\item Potassium Feldspar, not as abundant as plagioclase, but still extremely abundant and common mineral.
\item Kind of like quartz.
\item In general, you need silica-rich Environments, \ce{KAlSi3}, that's a lot of silica.
\item A felsic rock. Compositionally, we need that felisc composition to get Feldspar
\item very high temperature
\item these things are called granualites
\item metamorphic fomration
\item IN addition to the formula we just covered, often times in KSpar, we can have Fe$^{\text{3+}}$ subbing in.
\begin{itemize}
\item It would substitute for Aluminum. 1-for-1
\item when we get this sub in KSpar, it gives it a pink color.
\item most Kspars are a little pink/orange (not found in Plagioclase)
\item Another substitution is Lead - gives a green Amazonite
\end{itemize}
\end{itemize}
\item All feldspars are very common in sedementary rocks
\begin{itemize}
\item only thing more common is quartz.
\item quartz is just more resistent to weathering
\end{itemize}
\item An Arcose Sandstone
\begin{itemize}
\item A feldspar-rich sandstone. Fairly immature
\end{itemize}
\item Pegmatite
\begin{itemize}
\item All examples we saw in lab with beautiful crystals, those are from pegmatites
\item very water rich environment
\end{itemize}
\item Plagioclase Feldspar
\begin{itemize}
\item its everywhere, ther isn't a rock type you can't get plagioclase from
\item igneous, hydrothermal, .. it's everywhere
\item In more felisc rocks, things tend to be more sodium rich or Albite rich
\begin{itemize}
\item so granite or rhylite. Plagioclase there will be more albite
\end{itemize}
\item In more mafic rocks, more sodium rich
\begin{itemize}
\item basalt, gavrow, anorthite rich
\item Also generally correspond to tempratures
\item Boen's reaction series (higher T - Mafic, lower T Sodium rich, felsic)
\item Know the Feldspar Triangle
\end{itemize}
\end{itemize}
\item Techtosilicates
\end{itemize}
\subsection{Zeolite Group}
\label{sec-2-1}
\begin{itemize}
\item in terms of variety and \# of minerals. The zeolites contain the most different mineral varities of any silicate
\item 80 naturally occuring zeolites, more synth'd in a lab
\item All zeolites contain structural water. H2O
\item Two ways water can be held in minerals
\begin{itemize}
\item Structural water:
\begin{itemize}
\item Molecules of water that live inside the structure
\end{itemize}
\item Others have OH in their chemistry and get another H when disturbed.
\item Formula: M$_{\text{x}}$ D$_{\text{y}}$ Al. Formula page 1.
\item The ratio of Aluminum to Silica depends on the M-side. They tend to be mono-valent cations
\item D tends to be di-valent cations (Ca, Mn)
\item Structure itself: silicon tetrahedra with long, wide-open tunnels or holes where the water gets in.
\end{itemize}
\item (Na, Ca, H2O) are loosly bound and exchangable in the structure.
\begin{itemize}
\item these make it absorbant
\item Dessecant
\item Na+ -><- Ca$^{\text{2+}}$
\end{itemize}
\item Mineral Use:
\begin{itemize}
\item 
\end{itemize}
\item Common Zeolites: (not that important, recognize them as zeolites)
\begin{itemize}
\item Amalcine (Na) Isometric (looks like Leucite, but different size and shape)
\item Chabazite (Ca, Na, u) Triclinic (mixed sort of mineral, triclinic. Looks hexagonal (pseudohexagonal))
\item Heulandite (Na, Ca) Monoclinic
\item Stillbite (Na, Ca) Monoclinic (elongaded, blady sheets)
\item Naturolite (Na) Orthorhombic (radiating fibres)
\end{itemize}
\item Where do zeolites form?
\begin{itemize}
\item Environment is Low-Low Temperature Metamorphism.
\item Hydrothermal Alteration
\item common rocks with a lot of K, Na, Ca (like in feldspar)
\item \textbf{Low Temp Alteration \& metamorphism, often in mafic rocks}
\item Mafic rocks are very susceptable to low T, hydrothermal alteration
\item They break down and make zeolites.
\item Often fine-grained fillings in cracks or voids, or vesicles.
\item Great place for Zeolites to form.
\item Difficult to identify these fine grains without going to XRD.
\end{itemize}
\end{itemize}


silica group
feldspar group
zeolite group
\subsection{Category (Sodalite, Leucite, Nephiline): Feldspathoids}
\label{sec-2-2}

\begin{itemize}
\item Like feldspar, but they exist in Low-Si rocks.
\begin{itemize}
\item "Silica-Undersaturated environments"
\item Not enough Si to form Quartz
\item Still have Alkali Elements, those are necessary for Feldspathoids
\item Weird to have that kind of environment.
\end{itemize}
\end{itemize}


\begin{itemize}
\item Leucite: \ce{KAlSi2O6}
\begin{itemize}
\item soccer ball-shaped, roundish crystals
\item Formula looks a lot like KSpar (KAl)
\item Why would we get Leucite instead of KSpar?
\begin{itemize}
\item <write reaction>
\item <Look for Pics>
\item If we low silica rocks/highSilica Rocks
\item lookup (30:00):question:
\end{itemize}
\end{itemize}
\item Nepheline <complicated formula>
\begin{itemize}
\item Formula for Nepheline
\item Has a little sodium and calcium, not so pure in real world
\item One other Mineral End Membe: Nepheline has complete solid solution to
\begin{itemize}
\item KAlSiO4 (Calcilite), the end of a solid solution.
\end{itemize}
\end{itemize}
\item Bonus:
\begin{itemize}
\item Two common igneous rock types. Silica undersaturated and have abundant nephaline
\item Phonalite (igneous rocks)
\begin{itemize}
\item pings when you hit it, doesn't break either
\item Si-undersaturated
\item Alkali-enriched
\item Volcanic rock
\end{itemize}
\item Syenite (the intrusive rock)
\begin{itemize}
\item silica undersaturated, intrusive rock
\end{itemize}
\end{itemize}
\item Other Feldspathoids (foids):
\begin{itemize}
\item Sodalite: Na8(AlSiO4)6 Cl2 (chlorine, otherwise it's albite)
\item Lazurite:
\begin{itemize}
\item (Na,Ca)8 (AlSiO4)6 (SO4, Cl, S)2
\item Beautiful Azure Blue color
\item Primary constituant of Lapiz Lazuli
\end{itemize}
\item Petalite: Li(AlSi4O10)
\begin{itemize}
\item society really likes Petalite for it's lithium. We <3 Lithium
\item Hard to find large quantities of Lithium
\item This can be found in Lithium-Rich Pegmatites
\item :question: why does Petalite buck-trends. why unusual?
\end{itemize}
\end{itemize}
\end{itemize}

\subsection{Scapolite (it's own little group)}
\label{sec-2-3}
\begin{itemize}
\item solid solution between two end members
\begin{itemize}
\item (Marialite) Na4Al3Si9O24Cl
\item (Meionite) Ca4Al6Si6O24(CO3)
\item So, bring in Na, Ca
\begin{itemize}
\item Except for the Cl and the CO3 (carbonate), these look like:
\begin{itemize}
\item very common mineral things are identical to\ldots{} Plagioclase Feldspars! Anorthite, Albite
\end{itemize}
\item Scapolite takes the place of Plagioclase Feldspars in a \textbf{contact-metamorphic environment}
\item Magma intrusions cause magma intrusions. (Silicate Magma's into a carbonate limestone)
\item CO2 will be burned off into the atmosphere
\item That's where Scapolite forms. It's an indicator of these conditions.
\end{itemize}
\end{itemize}
\item Keep End-Members Of Solid Solution Paired
\begin{itemize}
\item Mariolite, Myanite
\item (Feldspars): Albite, Anorthite
\end{itemize}
\end{itemize}
END OF TECHTOSILICATES
\texttt{==============================================================================}

\section{Phyllosilicates}
\label{sec-3}
\begin{itemize}
\item Slightly less polymerizatoin of silica
\item Construct Phyllosilicates
\item There are fundamental structures of phyllosiliates
\item \textbf{Know the Structure First}
\item they're sheet silicates
\item Phyllo dough, sheety. fun fact.
\item Silicon tetrahedra shares 3 of its oxygens with other silicon tetrahedra. Leaving 1 to bond with other things
\item Phyllosilicates are generally soft (1's and 2's) :important:
\item Not very dense, fairly open structures () :important:
\item Single plane of leavage, easy to identify cleavage :important:
\end{itemize}
\subsection{Structure}
\label{sec-3-1}
\begin{itemize}
\item Combination of Tetrahedral Sheets (T)
\item and Octahedral Sheets (O)
\item Tetrahedral Sheets:
\begin{itemize}
\item Tetrahedral Sites, connected in 2-D Sheet
\item filled with either Al, or Si.
\end{itemize}
\item Octahedral Sheets:
\begin{itemize}
\item Made out of 2 adjacent planes of OH groups.
\item Between which, are a whole bunch of Octahedral, edge-sharing, sites.
\end{itemize}
\item Between these two OH groups, we get our octahedral sites.
\begin{itemize}
\item Filled with Cations
\item Depending on the Cations, there are 2 types of Octahedral Sheets
\begin{itemize}
\item Tri-Octahedral Sheet:
\begin{itemize}
\item 3/3 sites are filled with a 2+ cation (Mg, Fe)
\end{itemize}
\item Di-Octahedral Sheet:
\begin{itemize}
\item 2 out of 3 sites are filled with 3+ Cation (Al)
\item in the other site, it's vacant
\end{itemize}
\end{itemize}
\end{itemize}
\item These building blocks are important:
\begin{enumerate}
\item Tetrahedral Sheets
\item Tri-Octahedral Sheets
\item Di-Octahedral Sheets
\end{enumerate}
\item Now we glue the different sheets together:
\begin{itemize}
\item Combine T and/or O sheets to make Layers
\item Two Most Fundamental (T is trapezoid) (O is rectangle)
\begin{itemize}
\item (T)(O) Layer
\item (T)(O)(T) Layer
\item (O)\ldots{} sometimes\ldots{}
\end{itemize}
\item Layers are bound together by VanDerWaals or Interlayer Cation
\end{itemize}
\item TABLEL CONSTRUCTION
\end{itemize}

\begin{center}
\begin{tabular}{llll}
DiOctahedral & Diagram & Tri-Octahedral & Notes\\
\hline
\ce{Al2(OH)6} & (O) & \ce{Mg3(OH)6} & These aren't silicates\\
Gibsite &  & Brucite & they're hydroxides!\\
(ore for Al) &  &  & \\
\hline
\ce{Al2Si2O5(OH)4} & (T) & \ce{Mg3(Si2O5)(OH)2} & \\
Kaolinite & (O) & Serpentine & \\
(most abundant Clay) &  & swap Al for Mg & \\
\hline
\ce{Al2Si4O10(OH)2} & (T) & \ce{Mg3Si4O10(OH)2} & \\
Pyrophylite & (O) & Talc & \\
w/ VanDerWaals & (T) &  & \\
\hline
\ce{KAl2(AlSi3)O10(OH)2} & + & \ce{KMg3(AlSi3)O10(OH)2} & \\
Muscovite (kendmember) & (T) & Biotite & \\
\ce{NaAl2(AlSi3)O11(OH)2} & (O) &  & MICAS: (TOT)\\
Paraganite & (T) &  & w/ interlayer\\
 & + &  & cation\\
\hline
\ce{Ca(Al2)(Al2Si2)O10(OH)2} & 2+ &  & \\
Margarite & (T) &  & \\
 & (O) &  & \\
 & (T) &  & \\
 & 2+ &  & \\
\hline
 & (Talc) &  & \\
 & (Brucite) & Chloride & \\
 & (Talc) & 3rd most common & \\
 &  & phyllosilicate & \\
\end{tabular}
\end{center}

:question: about K end member in the table


Look at page 5 for the phengite information\ldots{} Ask :question: about it.

\section{Phyllosilicates Continued \textit{<2016-02-25 Thu>}}
\label{sec-4}
\begin{itemize}
\item Phyllosilicates cont.
\item Phengite
\begin{itemize}
\item similar to muscovite, but it takes the place of muscovite at High Pressures
\item \ce{K(Al2)(AlSi3)O10(OH)2} (Muscovite)
\begin{itemize}
\item Add some Mg$^{\text{2+}}$ and some Si for Al and creates Phengite
\item (4si/4Al) = (3/1)
\item Phengite is a really common indicator of ultra-high pressure/conditions
\end{itemize}
\end{itemize}
\item Paragonite is muscovite except remove K and put in Na
\end{itemize}
\subsection{Quick Review and Finish Up}
\label{sec-4-1}
\subsubsection{Micas}
\label{sec-4-1-1}
\begin{itemize}
\item TOT + interlayer Cations
\item Biotite:
\begin{itemize}
\item \textbf{2 end members (Fe$^{\text{2+}}$ Anite), (Mg$^{\text{2+}}$ Phlogopite)}
\item Common Brown Mica, can get it everywhere
\item Need a lil Mg and Fe and a Na
\end{itemize}
\item Muscovite (one end member) :question:
\begin{itemize}
\item Silvery/clear
\item restricted to felsic, igneous, metamorphic, and sedementary derivatives
\end{itemize}
\item Paragonite:
\begin{itemize}
\item Na-rich
\item Higher Pressure Mica
\item Paragonite, like Phengite, is an indicator of High Pressure
\item sometimes, paragonite is more white than muscovite's silvery whitish color
\end{itemize}
\item Phengite:
\begin{itemize}
\item Si-rich, high pressure
\end{itemize}
\item Muscovite, Paragonite, Phengite can be called "White Mica"
\item Glauconite:
\begin{itemize}
\item green mica, because Fe$^{\text{3+}}$
\item Often fine grained in marine sedimentsf
\end{itemize}
\end{itemize}
END MICAS

\rule{\linewidth}{0.5pt}
\subsection{Other Phyllosilicates}
\label{sec-4-2}
\begin{itemize}
\item Chlorite:
\begin{itemize}
\item Dark Green
\item low-temperature metamorphic rocks
\item hydrating mafic protoliths
\item Mafic Protoligh (basalt)
\item Green Schist/Green Stone is green becasue of chlorite
\item looks like a mica
\item forms with the micas, muscovite \& friends
\item How to tell Chlorite from others Micas ?:
\begin{itemize}
\item Micas are Elastic, Chlorite isn't. (+ color)
\end{itemize}
\end{itemize}
\item Talc:
\begin{itemize}
\item soft, greasy, fun to look at
\item Common in hydration of mafic to UltraMafic rocks
\item Hydration, Low-T metamorphism of ultramafic rocks
\item Water added to Basalt, but you need "Ultra"mafic protolith such as:
\begin{itemize}
\item the mantle!
\end{itemize}
\item Inject water into the mantle and you'll make things like Talc
\item Or Serpentine!!
\end{itemize}
\item Serpentine
\begin{itemize}
\item Common in metamorphic hydration (add water to mantle, will make Talc, Brucite, Serpentine)
\item Is a fun one:)
\item \uline{Very water rich}. It has an incredible solid storehouse of water. Up to 15\% weight in water.
\item 3 Different Crystal forms Serpentine can take :important:
\item They're serpentine chemicallly and structurally.
\item Serpentine is a (T)(O) structure. But it can bend to fit things together. Bent Sheets (see page 1).
\begin{itemize}
\item Lizardite: random ordering of bent TO's. Massive texture.
\item Antigorite: Ordered, wavy formation.
\item Crysotile: Ordered, in a circle. That's where we get the needles. Sheet Needle.
\begin{itemize}
\item Asbestos form of Crysotile, and of Serpentine
\item It's not structurally a needle, it's a sheet.
\item not as bad as amphibole asbestos
\end{itemize}
\end{itemize}
\end{itemize}
\end{itemize}
\subsection{Clay Minerals}
\label{sec-4-3}
\begin{itemize}
\item XRD needed to identify the Clay Minerals
\item sedementology - "clay" is anything finer than a grain size :question:
\item mineralology - "clay" is a phylosilicate tends to be fine grained.
\item Two Broad categories of Clay Minerals
\end{itemize}
\subsubsection{1:1 Clays}
\label{sec-4-3-1}
(T)(O) 
ex: Kaolinite
\begin{itemize}
\item strictly speaking, serpentine counts here. But it doesn't form in the same environment so we don't count it.
\end{itemize}
\subsubsection{2:1 Clays}
\label{sec-4-3-2}
(T)(O)(T)
ex: (\textbf{don't have their own formula, know that they're 2:1 clays, and that Smectite, Verm, and Montmor. form due to weathering of Mafic Rocks have a property of shrinking and swelling due to water uptake in the interlayer regions})
\begin{itemize}
\item Smectite (includes some mix of Ca, Mg, Fe in the octahedral sites)
\item Vermiculite
\item Montmorillionite
\item Illite (brings in K, similar to muscovite)
\item \textbf{Kaolinite and Illite form from weathering of Felsic Rocks}
\begin{itemize}
\item that's a big deal
\end{itemize}
\item We could add Glauconite (K, Fe$^{\text{3+}}$) to the mix.
\begin{itemize}
\item Often forms as a Clay in marine, sedimentary environments
\item sediment binds them
\end{itemize}
\item \textbf{Smec, Verm, Montor. they expand and dessicate with water. By like 50\%!!! That's huge}
\item \textbf{Because they're so fine grained, you can't identify these potential dangers unless you resort to XRD}
\end{itemize}
END OF PHYLLOSILICATES
\texttt{==============================================================================}
\section{Inosilicates}
\label{sec-5}
\begin{itemize}
\item Two types:
\begin{itemize}
\item single chain silicates = pyroxenes. pyroxenes are single chain silicates. the end.
\begin{itemize}
\item End in \ce{SiO3}
\end{itemize}
\item Dobule chiain silicates = amphiboles.
\begin{itemize}
\item End in \ce{Si8O22}
\end{itemize}
\end{itemize}
\end{itemize}
\subsection{Pyroxenes}
\label{sec-5-1}
\begin{itemize}
\item LEARN THE STRUCTURE FIRST!!!, just like with the phyllosilicates
\item And amphibole can get ugly formula-wise, so learn by structure first
\item like feldspar with triangle, learn the triangle and use structure
\end{itemize}
\subsubsection{Structure of Pyroxenes}
\label{sec-5-1-1}
\begin{itemize}
\item silicon tetrahedron's sitting flat on the board, stacked end to end. C-axis. Page 2.
\item Along C-Axis view: $\Delta$$\Delta$$\Delta$
\end{itemize}
% Emacs 24.4.1 (Org mode 8.2.10)
\end{document}